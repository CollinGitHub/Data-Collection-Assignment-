\documentclass[]{article}

%opening
\title{DATA COLLECTION CONCEPT ON
	\linebreak
	{EATING PLACES ON MAKERERE UNIVERSITY KAMPALA CAMPUS}
}
\author{KAJUBI COLLIN NUWA 14/U/314 214000077}

\begin{document}

\maketitle\centering{BACKGROUND}

Makerere University is a huge university that boasts of a\hfill population of around 50,000 students. These students taking part is hundreds of tertiary education programs across the university. Some of the students reside on the campus but most reside outside the campus. Some come from home, others sleep in hostels, rentals and a few in halls located on campus. But these students all have one thing in common, they all study on the same campus.
\linebreak 
\linebreak
Food is one of the four basic needs of a human being and to feed a dense population of around 50,000 people, in this case Makerere University students is a task that would challenge a single entity. So to achieve this task, it takes is a joint effort of a good number of companies. These companies provide a wide variety of food ranging from snacks to heavy healthy meals.
\linebreak
\linebreak



\section*{PROBLEM STATEMENT}}
The purpose of this data collection concept is to shade some light on the companies taking part in this task through collecting some data on them. As the task of providing food for a population of 50,000 students is not a simple task.


\section*{OBJECTIVES}
1)	To set up and Open Data Kit Aggregate Server to upload the forms for data collection.

2)	To create forms using open data kit to use to collect the data from the eating places. 

3)	To collect various data on the eating places that includes; the name of the eating place, the type of eating place either a restaurant or a canteen, the location of the eating place, the picture of the eating place, the opening and closing times.

4)	To analyze the data and draw out a meaningful conclusion.

\section*{SCOPE}
The geographical scope is the Makerere University Campus i.e eating places such as restaurants and canteens on the Makerere University Campus.
\section*{SIGNIFICANCE}
The results of data collection and analysis can be used in the future planning for providing food for the many students of Makerere University.
The data can also be used to give an insight to one planning to set up an eating place on the Makerere University Campus. This can be inform on what will be required to be competitive and provide good services, what times are ideal for opening and closing, what location is favourable or has less or no competitor, etc…







\end{document}
